%!TEX root = ./main-part-b-2.tex

\subsubsection{GWT}

\noindent
%

\paragraph{Role in the project.} 
GWT will lead ... which targets the creation of a secure container architecture.
GWT will coordinate the \proj project and hence, leads WP7 Project Management.

\smallskip
\noindent
\paragraph{Key personnel.}

\paragraph{Prof. Dr. Christof Fetzer (M)} has received his diploma in Computer Science from the University of Kaiserlautern, Germany (Dec. 1992) and his Ph.D. from UC San Diego (March 1997). 
As a student he received a two-year scholarship from the DAAD and won two best student paper awards (SRDS and DSN). 
He was a finalist of the 1998 Council of Graduate Schools/UMI distinguished dissertation award and received an IEE mather premium in 1999. Dr. Fetzer joined AT\&T Labs-Research in August 1999 
and had been a principal member of technical staff until March 2004. 
Since April 2004 he heads the endowed chair (Heinz-Nixdorf endowment) in Systems Engineering in the Computer Science Department at TU Dresden. 
He is the chair of the Distributed Systems Engineering International Masters Program at the Computer Science Department. 
Prof. Dr. Fetzer has published over 150 research papers in the field of dependable distributed systems, has been member of more than 50 program committees, has won three best paper awards (DEBS2013, LISA2013, SRDS2014),   his PhD students have won two best student paper awards (IEEE Cloud 2014, DSN2015),  and the EuroSys Roger Needham Award 2014.

\smallskip
\noindent
\paragraph{Andr{\'e} Martin (M)} graduated with a Diploma (2008) and a PhD in Computer Science (2015).
Since January 2016, he is a post doctoral researcher at the Systems Engineering chair. 
In his doctoral thesis, he explored novel mechanisms for low overhead fault tolerance in data streaming systems.
He has been selected twice as a DEBS challenge finalist in 2014 and 2015 and won the UCC Cloud Challenge award in 2014.
His expertise lies in cloud computing, distributed systems, elasticity and fault tolerance in large scale data processing systems.

\smallskip
\noindent
\paragraph{Dr. Irina Karadschow (F)} \smallskip
\noindent

\paragraph{Relevant publications.}
%% {\bf Probably something more recent here, eg, EuroSys 2016, Middleware 2015, SRDS 2015, DSN 2015, NSDI 2015, ATC 2014}

\begin{itemize}

\item OSDI Paper

\item Middleware paper 1

\item Middleware paper 2

\item Dmitrii Kuvaiskii, Rasha Faqeh, Pramod Bhatotia, Pascal Felber, Christof Fetzer. \emph{HAFT: Hardware-Assisted Fault Tolerance},
In Proceedings of the European Conference on Computer Systems (EuroSys), 2016

\item Thomas Knauth, Christof Fetzer. \emph{VeCycle: Recycling VM Checkpoints for Faster Migrations}, In Middleware, 2015

\item Dmitrii Kuvaiskii, Christof Fetzer. \emph{$\delta$-encoding: Practical Encoded Processing},
In Proceedings of The 45th Annual IEEE/IFIP International Conference on Dependable Systems and Networks (DSN), 2015, (Carter Award)

\item Andr{\'e} Martin, Tiaraju Smaneoto, Tobias Dietze, Andrey Brito, Christof Fetzer.
\emph{User-Constraint and Self-Adaptive Fault Tolerance for Event Stream Processing Systems},
In Proceedings of The 45th Annual IEEE/IFIP International Conference on Dependable Systems and Networks (DSN), 2015

\item Diogo Behrens, Marco Serafini, Flavio P. Junqueira, Sergei Arnautov, Christof Fetzer.
\emph{Scalable Error Isolation for Distributed Systems},
In 12th USENIX Symposium on Networked Systems Design and Implementation (NSDI), 2015

\item  Diogo Behrens, Dmitrii Kuvaiskii, Christof Fetzer. \emph{HardPaxos: Replication Hardened Against Hardware Errors},
In Proceedings of the 33rd IEEE Symposium on Reliable Distributed Systems (SRDS), 2014 (best paper award)

\end{itemize}
 
\smallskip
\noindent
\paragraph{Relevant experience.}
Prof. Dr Fetzer successfully participated in several EU-funded projects, including
VELOX\footnote{\url{http://www.velox-project.eu}},
STREAM\footnote{\url{http://www.streamproject.eu}},
SRT-15\footnote{\url{http://www.srt-15.eu}},
LEADS\footnote{\url{http://www.leads-project.eu}}, and
ParaDIME\footnote{\url{http://www.paradime.eu}}.
Moreover, the current research focus of Prof. Fetzer and his research group is on security in cloud environments.
Prof. Fetzer coordinates two EU H2020 research projects \textbf{SERECA\footnote{\url{http://www.serecaproject.eu}}}
and \textbf{SecureCloud\footnote{\url{http://www.securecloud-project.eu}}}). % that target approaches that allow secure execution of an application in non-trusted and potentially malicious environments. 
In the context of \textbf{SERECA}, the group developed initial software infrastructure for reactive applications to use novel 
trusted execution environments to preserve confidentiality and integrity of applications and data in public clouds while in the \textbf{SecureCloud} project,
the group extended the open-source IaaS platform OpenStack to support trusted computing in public clouds using Intel's SGX technology.
Prof. Fetzer’s group has 10 years of expertise in dependable and distributed systems, event-based communication systems, and cloud computing. 
This experience and knowledge will be integrated in \proj to achieve reliability and security of user applications and data processed in cloud-based infrastructures.
Through its extensive experience in project co-ordination and project management at national and international level GWT is well placed to take over the project coordination. 

\smallskip
\noindent
\paragraph{Infrastructures.}


\smallskip
\noindent
